\documentclass[a4paper,twoside,12pt]{book}

\usepackage{amsmath}
\usepackage{amstext}
\usepackage{amssymb}
\usepackage{amsfonts}
\usepackage{amsthm}
\usepackage{etoolbox}
%mathspec/fontspec
\usepackage{mathspec}
\usepackage{xunicode}
\usepackage{xltxtra}

\usepackage[bookmarks,bookmarksnumbered]{hyperref}

\makeatletter       % changes [1] to 1. in bibliography
\renewcommand{\@biblabel}[1]{#1.}
\makeatother

\usepackage{appendix}
\usepackage{graphicx}
\usepackage{natbib}


% FONT
\setmainfont[Mapping=tex-text,Numbers={Lining}]{Linux Libertine O}
\setmathfont(Latin)[Numbers={Lowercase}]{Linux Libertine O}
\setmathfont(Digits)[Scale=MatchLowercase]{Linux Libertine O}

\usepackage{unicode-math}
\setmainfont[Mapping=tex-text,Numbers={Lining}]{Linux Libertine O} 
\setmathfont{XITS Math}

\setmathfont[range={\mathup}]{Linux Libertine O} 
\setmathfont[range={\mathit}]{Linux Libertine O Italic} 
\setmathfont[range={\rm}]{Linux Libertine O} 
\setmathfont[range={\mathrm}]{Linux Libertine O} 



% \renewcommand{\rmdefault}{fxl}
\renewcommand{\sfdefault}{txss}


%%%%%%%%% from here yuasa setting %%%%%%%%%%%%%%%%%%%%%
% a4paper paperheight = 845.04684pt
%\setlength{\topmargin}{-.5in}
%% textheight = paperheight - 2in - \topmargin - \headheight - \headsep
%%              - \footskip = approx. 641pt
%\setlength{\textheight}{1.05\paperheight}
%\addtolength{\textheight}{-2.2in}
%%\addtolength{\textheight}{-\topmargin}
%%\addtolength{\textheight}{-\headheight}
%%\addtolength{\textheight}{-\headsep}
%%\addtolength{\textheight}{-\footskip}

%% a4paper paperwidth = 597.50787pt
%%\setlength{\marginparsep}{0pt}
%%\setlength{\marginparwidth}{0pt}
%% textwidth = paperwidth - 2in - \oddsidemargin = approx. 432pt
%\setlength{\textwidth}{\paperwidth}
%\addtolength{\textwidth}{-1.8in}
%%\addtolength{\textwidth}{-\oddsidemargin}

%% add 20pt to sidemargin
%%   oddside: 1in+20pt+20pt | \textwidth | 1in-20pt
%%   evenside: 1in-20pt | \textwidth | 1in+20pt+20pt
%\setlength{\oddsidemargin}{0pt}
%\setlength{\evensidemargin}{0pt}

%%%%%%%%% to here yuasa setting %%%%%%%%%%%%%%%%

%%%%%%%%% from here margin setting %%%%%%%%%%%%%%%%
%for dron
%\setlength{\topmargin}{0pt}
%\setlength{\marginparsep}{0pt}
%\setlength{\marginparwidth}{0pt}
%\setlength{\oddsidemargin}{20pt}
%\setlength{\evensidemargin}{0pt}
%
%\setlength{\textwidth}{\paperwidth}
%\addtolength{\textwidth}{-2in}
%\addtolength{\textwidth}{-\oddsidemargin}
%
%\addtolength{\oddsidemargin}{0pt}
%\addtolength{\evensidemargin}{-23pt}
%\setlength{\topmargin}{-12mm}
%\setlength{\textheight}{25.3cm}
%\setlength{\textwidth}{16.3cm}
%%%%%%%%% till here margin setting %%%%%%%%%%%%%%%%

%%% Linestrech
%for user guide
\renewcommand{\baselinestretch} {1.1}
%for dron
%\renewcommand{\baselinestretch} {1.1}

\setlength\bibsep{2pt}
\renewcommand{\bibnumfmt}[1]{(#1)}


%%% Table of Contents
\usepackage{tocloft}
\setlength\cftparskip{-0.7pt}
\setlength\cftbeforesecskip{1.5pt}
\setlength\cftaftertoctitleskip{2pt}

\addtolength{\oddsidemargin}{-.875in}
	\addtolength{\evensidemargin}{-.875in}
	\addtolength{\textwidth}{1.75in}

	\addtolength{\topmargin}{-.875in}
	\addtolength{\textheight}{1.75in}

%% for quotes
%\fontspec[Mapping=tex-text]{Linux Libertine O}
%\fontspec[Ligatures=TeX]{Linux Libertine O}

\title{\Huge{
SpaceWire-to-GigabitEther\\
and SpaceWire RMAP GUI}\\
\Large{User Guide and Tutorial
}}

\author{
{\Large
Takayuki Yuasa
}\\
{\small Japan Aerospace Exploration Agency, Institute of Space and Astronautical Science}\\
}

\date{December, 2011}


\begin{document}
\maketitle

\setcounter{page}{2}


\chapter{Overview}
\section{SpaceWire-to-GigabitEther}
\section{SpaceWire RMAP GUI}
\section{This is an open-source user guide}
\section{Feedback}
\section{Revision}

\chapter{SpaceWire-to-GigabitEther}
\section{Purpose}
\subsection{Standalone version and AMC version}
\subsection{Specification}
\subsection{Dimensions}
\subsection{Weight}
\subsection{Power}
\section{Hardware block diagram}
\section{FPGA block diagram}
\section{Hardware implementation of a TCP/IP stack}
\section{Configuration of the internal SpaceWire router}
\section{Physical and internal SpaceWire connections}

\chapter{The internal SpaceWire Router}
\section{Basic design}
\section{Configurable items}
\subsection{Default parameters}
\subsection{Typical usages}
\section{Router Configuration Port (Port0)}
\section{Register Port (Port??)}
\section{Internal loop connection}

\chapter{Performance}
\section{SpaceWire packet transfer}
\section{RMAP data transfer}
\section{Performance of the VHDL TCP/IP stack}

\chapter{SpaceWire RMAP GUI}
\section{Overview}
This software is an open-source project.
No warranty, and provided as is.
\section{SpaceWire I/F control section}
\section{SpaceWire-layer operation}
\section{RMAP-layer operation}
\section{Define RMAP target nodes using an XML file}
\section{Define registers on RMAP target nodes using an XML file}
\section{RMAP Packet Utility}
\section{Scripting}

\chapter{Tutorial}
\section{Send/receive SpaceWire packets}
\section{Read/write access to an RMAP target node}
\section{Access the configuration port of the internal SpaceWire router}
\subsection{Access the routing table}
\subsection{Change link operating frequencies of SpaceWire ports}

\appendix
\chapter{TCP/IP-SpaceWire packet transfer protocol}
\chapter{Onboard CPU for configuration}
SpaceWire-to-GigabitEther implements a V850 processor for configuration via TCP/IP. 
\chapter{The SpaceWire/RMAP Library}

\section{Basic design}





\end{document}